\documentclass{article}

\usepackage[utf8]{inputenc} 

\usepackage{amsmath, amssymb, amsthm}
\usepackage{mathtools}
\usepackage{graphicx}
\usepackage{hyperref}
\usepackage{enumitem}
\usepackage{authblk}
\usepackage{bbm}
\usepackage[margin=1in]{geometry}

\newcommand{\GAAOdef}{GAAO 1.0 (General Adaptive Agent Ontology), a unified event-sourced ontology for modeling and steering adaptive agents through time, integrating semantic topology, constraint systems, condition models, evidential reasoning, transformation dynamics, and recursive adaptation loops}

\title{\GAAOdef}
\author[1]{Robert Swaab}
\affil[1]{Independent Researcher}
\date{December 9, 2025}

\begin{document}
\maketitle

\begin{abstract}
We present \GAAOdef. 
GAAO formalizes a general-purpose adaptive agent as the tuple 
$A = (E, C, K, X, R, P, \Omega, I, L)$,
where the agent is defined by an event-sourced ledger of behaviour, a hierarchical semantic topology, an explicit constraint fabric, a structured condition space, an evidential graph, a transformation layer capturing progress metrics and outcome effects, an adaptive reasoning engine, and a recursive adaptation loop governing temporal evolution of state.
Unlike existing paradigms such as BDI, ACT-R, reinforcement learning, MAPE-K, or event-sourced architectures, GAAO unifies semantics, constraints, conditions, evidence, transformation dynamics, and recursive feedback into a single ontology for modeling and steering adaptive behaviour over time.
\end{abstract}

\section{Introduction}

Adaptive agents---biological, artificial, organisational, or hybrid---must continuously interpret conditions, choose actions, accumulate evidence, and adjust behaviour across time.
Most existing architectures focus on a subset of this problem:

\begin{itemize}
    \item BDI: beliefs, desires, intentions and plan choice,
    \item ACT-R: human cognitive processes via production rules and memory systems,
    \item reinforcement learning (RL): reward maximisation in Markovian environments,
    \item MAPE-K: monitor--analyse--plan--execute over a knowledge base,
    \item event-sourced architectures: state reconstruction from event streams.
\end{itemize}

What is missing is a single ontology that simultaneously:

\begin{enumerate}
    \item treats behaviour as an event-sourced ledger over time,
    \item situates behaviour in a semantic topology of containers,
    \item represents constraints as explicit commitments with evaluation functions,
    \item models conditions as a structured space of dimensions and models,
    \item treats evidence, deltas, drift, and trajectories as first-class objects,
    \item captures progress and outcomes as transformation dynamics,
    \item and closes the loop with a recursive adaptation operator over agent state.
\end{enumerate}

GAAO is proposed as such an ontology.
It is not a single algorithm or implementation, but a formal structure from which many architectures (including human life systems, software agents, organisational agents, and hybrid human--AI systems) can be instantiated.

In what follows we:

\begin{itemize}
    \item formally define the GAAO tuple and its layers,
    \item provide a mathematical specification,
    \item compare GAAO dimensionally with existing architectures (BDI, ACT-R, RL, MAPE-K, event-sourced systems),
    \item and synthesise the uniqueness of GAAO as a unifying ontology for adaptive agents.
\end{itemize}

\section{Symbol Table}

Table~\ref{tab:symbols} summarises the primary symbols used in GAAO.

\begin{table}[h]
\centering
\begin{tabular}{ll}
\hline
Symbol & Meaning \\
\hline
$T$ & Time domain \\
$A$ & Attribute space \\
$M$ & Engagement mode set \\
$C$ & Semantic containers \\
$\mathcal{T}$ & Container-type set \\
$K$ & Constraints \\
$\mathcal{P}(K)$ & Power set of constraints $K$ \\
$X_d$ & Condition dimensions \\
$X_m$ & Condition models \\
$X$ & Condition space ($X_d \cup X_m$) \\
$R$ & Evidence records \\
$P$ & Progress records \\
$\Omega$ & Outcome records \\
$D$ & Deviation space \\
$\Lambda$ & Metric space (for progress) \\
$\mathcal{T}_r$ & Trajectory extraction operator \\
$\mathrm{Drift}$ & Drift extraction operator \\
$\Phi$ & Pattern space (drift and trajectories) \\
$\Psi$ & Deviation-classification space \\
$L$ & Recursive adaptation loop schema \\
$\mathcal{L}$ & State-transition operator induced by $L$ \\
\hline
\end{tabular}
\caption{Symbol table for GAAO.}
\label{tab:symbols}
\end{table}

\section{Formal Definition of the General Adaptive Agent Ontology}

A General Adaptive Agent in GAAO is defined as:
\begin{equation}
    A = (E, C, K, X, R, P, \Omega, I, L),
\end{equation}
where:
\begin{itemize}[leftmargin=2cm]
    \item[$E$] Event Ledger Layer,
    \item[$C$] Semantic Topology Layer,
    \item[$K$] Constraint Fabric,
    \item[$X$] Condition Space Layer,
    \item[$R$] Evidential Graph Layer,
    \item[$P$] Progress Records (part of transformation layer),
    \item[$\Omega$] Outcome Records (part of transformation layer),
    \item[$I$] Adaptive Reasoning Engine,
    \item[$L$] Recursive Adaptation Loop (schema and induced operator).
\end{itemize}

We now define each component.

\subsection{Event Ledger Layer}

Let:
\[
T \text{ (time domain)}, \quad
A \text{ (attribute space)}, \quad
M \text{ (engagement mode set)}.
\]

An event is defined as:
\begin{equation}
e = (t_s, t_e, \lambda_c, \lambda_m, \pi, \alpha, \sigma, \omega, \delta, \kappa),
\end{equation}
where:
\begin{align*}
    t_s, t_e &\in T & \text{start and end times}, \\
    \lambda_c &\in C & \text{semantic container identifier}, \\
    \lambda_m &\in M & \text{engagement mode}, \\
    \pi, \alpha &\in A & \text{planned and actual attributes}, \\
    \sigma &\in X & \text{condition snapshot}, \\
    \omega &\in \Omega & \text{outcome record}, \\
    \delta &\in D & \text{deviation signal}, \\
    \kappa &\subseteq K & \text{applicable constraints}.
\end{align*}

The Event Ledger is the set:
\[
E = \{ e_1, e_2, \ldots \}.
\]
As a type, we can write:
\begin{equation}
E \subseteq T \times T \times C \times M \times A \times A \times X \times \Omega \times D \times \mathcal{P}(K),
\end{equation}
where $\mathcal{P}(K)$ is the power set of $K$.

\subsection{Semantic Topology Layer}

Let $\mathcal{T}$ be the set of container types.

A semantic container is:
\begin{equation}
c = (id, \tau, parent, \varsigma, H_e, H_\omega, H_p, H_\phi),
\end{equation}
where:
\begin{itemize}
    \item $id$ is a unique container identifier,
    \item $\tau \in \mathcal{T}$ is the container type,
    \item $parent \in C \cup \{\emptyset\}$ is the unique parent container or null (root),
    \item $\varsigma$ is a container state vector,
    \item $H_e$ is event history attached to the container,
    \item $H_\omega$ is outcome history,
    \item $H_p$ is progress history,
    \item $H_\phi$ is drift history (sequence of drift signals $\phi$).
\end{itemize}

The semantic topology is a rooted tree:
\[
G = (C, parent),
\]
with standard tree properties (one parent, acyclic, connected at the root).

\subsection{Constraint Fabric}

A constraint is:
\begin{equation}
k = (id, \iota, \theta, \mu, W, L_c, \gamma, H_k),
\end{equation}
where:
\begin{itemize}
    \item $\iota$ is an intention descriptor,
    \item $\theta$ is an obligation specification,
    \item $\mu$ is a measurement mode,
    \item $W \subseteq T$ is an activation window,
    \item $L_c \subseteq C$ is the set of bound containers,
    \item $\gamma$ is an evaluation function,
    \item $H_k$ is the constraint's adjustment history.
\end{itemize}

For a given constraint $k \in K$, the evaluation function is:
\begin{equation}
\gamma_k : (E, X, C) \to [0,1] \cup \{\text{fulfilled}, \text{violated}\}.
\end{equation}

\subsection{Transformation Layer: Progress and Outcomes}

Let $\Lambda$ be the metric space for progress measurement.

A progress record is:
\begin{equation}
p = (c, metric, v, d, e, t),
\end{equation}
where:
\begin{itemize}
    \item $c \in C$ is the container in which progress occurs,
    \item $metric \in \Lambda$ is the progress metric,
    \item $v$ is the magnitude,
    \item $d \in \{-1, 0, 1\}$ is the direction (away from, neutral, toward),
    \item $e \in E$ is the originating event,
    \item $t \in T$ is the timestamp.
\end{itemize}

An outcome record is:
\begin{equation}
\omega = (i, x, s, \delta),
\end{equation}
where:
\begin{itemize}
    \item $i$ captures internal effects,
    \item $x$ captures external effects,
    \item $s$ is a state-transition marker,
    \item $\delta \in D$ is a deviation classification.
\end{itemize}

\subsection{Condition Space Layer}

Let:
\[
X_d \text{ be the set of condition dimensions}, \quad
X_m \text{ be the set of condition models}, \quad
X = X_d \cup X_m.
\]

A condition dimension is:
\[
\xi = (name, type, value, t, source, conf, exp),
\]
capturing an atomic contextual or state-like quantity.

A condition model is:
\begin{equation}
M = (name, type, inputs, rules, update, confidence, version),
\end{equation}
where:
\begin{itemize}
    \item $inputs \subseteq X_d$,
    \item $rules$ encodes interpretive logic,
    \item $update : (X, R) \to X$ defines model evolution,
    \item $confidence$ is a model-level confidence score,
    \item $version$ is a revision identifier.
\end{itemize}

The condition profile at time $t$ is:
\[
X_t \subseteq X.
\]

\subsection{Evidential Graph Layer}

Let $D$ be the deviation space.

An evidence record is:
\begin{equation}
r = (id, type, t, c, k, e, raw, derived, conf, src),
\end{equation}
where $type$ may be one of:
\{EventEvidence, DeltaEvidence, ProgressEvidence, OutcomeEvidence, DriftSignal, TrajectorySignal\}.

A delta record is:
\begin{equation}
\delta = f(\pi, \alpha) \in D,
\end{equation}
computed from planned vs.\ actual attributes.

A drift signal is:
\begin{equation}
\phi = (type, magnitude, recurrence, link, t_1, t_n),
\end{equation}
and a trajectory signal is:
\begin{equation}
\tau = (pattern, container, direction, confidence, window).
\end{equation}

In addition to the delta operator $f_\delta$, GAAO defines evidential extraction operators:
\begin{equation}
\mathcal{T}_r : R \to \{\tau_1,\dots,\tau_m\},
\end{equation}
\begin{equation}
\mathrm{Drift} : R \to \{\phi_1,\dots,\phi_n\},
\end{equation}
mapping evidence records into recognised trajectory and drift-signal sets.

\subsection{Adaptive Reasoning Engine}

The adaptive reasoning engine is:
\begin{equation}
I : (X, E, K, C, R) \to \{\Pi, \Delta, S, \Upsilon\},
\end{equation}
where:
\begin{itemize}
    \item $\Pi$ is the space of plan proposals,
    \item $\Delta$ is the space of adjustments,
    \item $S$ is the space of simulations,
    \item $\Upsilon$ is the space of interpretive summaries.
\end{itemize}

Let:
\[
\Phi \text{ be the pattern space (drift and trajectories)}, \quad
\Psi \text{ be the deviation-classification space}.
\]

GAAO defines the following operators:
\begin{align*}
    I_{\text{int}}(R, X) &\to \Upsilon & \text{(interpretation)}, \\
    I_{\text{pat}}(R) &\to \Phi & \text{(pattern extraction)}, \\
    I_{\Delta} : D &\to \Psi & \text{(deviation classification)}, \\
    I_{\text{plan}}(K, X, C) &\to \Pi & \text{(planning)}, \\
    I_{\text{adj}}(K, C, X, R) &\to \Delta & \text{(adjustment)}, \\
    I_{\text{sim}}(state, window) &\to S & \text{(simulation)}.
\end{align*}

\subsection{Recursive Adaptation Loop}

The loop schema is:
\[
L = \{\text{Plan} \to \text{Execute} \to \text{Log} \to \text{Interpret} \to \text{Adjust}\}.
\]

The operational state at time $t$ is:
\[
state_t = (K_t, C_t, X_t, E_t, R_t).
\]

Let $\mathcal{L}$ be the induced state-transition operator:
\begin{equation}
    state_{t+1} = \mathcal{L}(state_t).
\end{equation}

Progress $P$ and outcomes $\Omega$ are derived from $(E, C, R)$ under this loop.

\section{Mathematical Specification Summary}

We briefly summarise the mathematical structure.

\subsection{Core Sets}

\begin{itemize}
    \item $T$ — time domain,
    \item $A$ — attribute space,
    \item $M$ — engagement modes,
    \item $C$ — containers,
    \item $\mathcal{T}$ — container-type set,
    \item $K$ — constraints,
    \item $X_d$ — condition dimensions,
    \item $X_m$ — condition models,
    \item $X = X_d \cup X_m$ — condition space,
    \item $R$ — evidence records,
    \item $P$ — progress records,
    \item $\Omega$ — outcome records,
    \item $D$ — deviation space,
    \item $\Lambda$ — metric space.
\end{itemize}

Events inhabit:
\begin{equation}
E \subseteq T \times T \times C \times M \times A \times A \times X \times \Omega \times D \times \mathcal{P}(K).
\end{equation}

The condition profile:
\[
X_t \subseteq X.
\]

Semantic topology:
\[
G = (C, parent), \quad parent : C \to C \cup \{\emptyset\}.
\]

Constraint evaluation:
\[
\gamma_k : (E, X, C) \to [0,1] \cup \{\text{fulfilled}, \text{violated}\}.
\]

Condition dynamics:
\[
update_M : (X, R) \to X, \quad update_X : (X, R) \to X.
\]

Evidential operators:
\[
f_\delta : ( \pi, \alpha ) \to D, \quad
\mathcal{T}_r : R \to \{\tau_1,\dots,\tau_m\}, \quad
\mathrm{Drift} : R \to \{\phi_1,\dots,\phi_n\}.
\]

Adaptive reasoning:
\[
I : (X, E, K, C, R) \to \{\Pi, \Delta, S, \Upsilon\}, \quad
I_{\Delta} : D \to \Psi.
\]

Recursive adaptation:
\[
state_t = (K_t, C_t, X_t, E_t, R_t), \quad state_{t+1} = \mathcal{L}(state_t).
\]

\section{Comparison to Existing Agent Architectures}

We sketch dimensional comparisons; these can be expanded in a longer version.

\subsection{BDI (Belief--Desire--Intention)}

BDI focuses on beliefs (world knowledge), desires (goals), intentions (committed goals), and plans.
In GAAO terms:

\begin{itemize}
    \item Beliefs correspond to aspects of $X$ and $R$,
    \item Desires correspond to intention descriptors $\iota$ in $K$,
    \item Intentions correspond to active constraints over windows $W$,
    \item Plans correspond to outputs of $I_{\text{plan}}$ and $I_{\text{sim}}$.
\end{itemize}

BDI captures a subset of GAAO's concerns: primarily goal/commitment management and deliberative planning.
GAAO additionally formalises evidential truth, delta and drift, semantic topology, and condition models as first-class components of the ontology.

\subsection{ACT-R}

ACT-R is a cognitive architecture for human cognition built on production rules and multiple memory systems.
GAAO differs by:

\begin{itemize}
    \item using an explicit event ledger $E$,
    \item modelling semantics via containers $C$ and types $\mathcal{T}$,
    \item treating constraints $K$ as commitments with evaluation functions,
    \item modelling conditions via $X_d, X_m$,
    \item using an evidential graph $R$ with drift and trajectories.
\end{itemize}

ACT-R provides a rich internal cognitive model, but does not attempt to be an event-sourced, constraint-governed, semantics-explicit ontology as GAAO does.

\subsection{Reinforcement Learning (RL)}

Reinforcement learning formalises:

\begin{itemize}
    \item state space,
    \item action space,
    \item reward function,
    \item policy and value functions.
\end{itemize}

In contrast:

\begin{itemize}
    \item RL optimises cumulative reward; GAAO maintains constraint alignment and adaptive behaviour under multiple signals.
    \item RL states are Markovian vectors; GAAO states include condition profiles $X_t$ and container states $\varsigma$.
    \item RL rewards are scalar; GAAO outcomes $\omega$ are multi-signal internal and external transformations.
    \item GAAO explicitly models semantics, constraints, evidence, and multi-scale adaptation.
\end{itemize}

\subsection{MAPE-K}

MAPE-K (monitor--analyse--plan--execute over knowledge) is a loop pattern for self-adaptive systems.
GAAO can be viewed as a granular refinement:

\begin{itemize}
    \item Monitor $\to$ Event Ledger + Evidential Graph $(E, R)$,
    \item Analyse $\to$ $I_{\text{int}}, I_{\text{pat}}, I_{\Delta}$ over $(X, R)$,
    \item Plan $\to$ $I_{\text{plan}}$,
    \item Execute $\to$ event realisation and updates to $E$,
    \item Knowledge $\to$ $(C, X, R)$ plus constraint fabric $K$.
\end{itemize}

\subsection{Event-Sourced Architectures}

Event-sourced systems store events as canonical truth and reconstruct state from event streams.
GAAO adopts this backbone but extends it with:

\begin{itemize}
    \item a semantic topology $C$,
    \item an explicit constraint fabric $K$,
    \item condition space $X$,
    \item evidential graph $R$,
    \item transformation layer $(P, \Omega)$,
    \item adaptive reasoning $I$,
    \item recursive adaptation loop $\mathcal{L}$.
\end{itemize}

\section{Synthesis of Uniqueness}

We summarise the structural differences of GAAO without overclaiming.

\subsection{Cross-Domain Integration}

GAAO integrates:

\begin{itemize}
    \item temporal event sourcing (Event Ledger $E$),
    \item hierarchical semantic structure (Semantic Topology $C$),
    \item formal constraint algebra (Constraint Fabric $K$),
    \item condition-space modelling (Condition Space $X$),
    \item evidential truth engine (Evidential Graph $R$ with deltas, drift, trajectories),
    \item transformation modelling (Progress $P$ and Outcomes $\Omega$),
    \item adaptive reasoning (engine $I$),
    \item recursive adaptation loop ($L$, $\mathcal{L}$).
\end{itemize}

Existing frameworks address subsets of this list.
GAAO's contribution is a unified ontology spanning all of them.

\subsection{Delta and Drift as First-Class Constructs}

GAAO treats:

\begin{itemize}
    \item $\delta = f(\pi, \alpha)$ (delta),
    \item drift signals $\phi$,
    \item trajectory signals $\tau$,
\end{itemize}

as first-class objects tightly coupled with constraints, containers, and trajectories.
This provides a principled way to reason about misalignment between intended and actual behaviour across time.

\subsection{Semantics, Evidence, Constraints Co-Defined}

GAAO co-defines:

\begin{itemize}
    \item where things happen (semantic containers $C$),
    \item what is supposed to happen (constraints $K$),
    \item what actually happens (events $E$ and evidence $R$),
    \item what changes (transformation layer $(P, \Omega)$),
    \item under what conditions it happens (condition space $X$),
    \item how behaviour is interpreted and adjusted (engine $I$ and loop $\mathcal{L}$).
\end{itemize}

This tight coupling of semantic structure, evidence, and constraints is not typical in existing agent or control architectures.

\subsection{Positioning Statement}

A cautious but accurate positioning is:

\begin{quote}
To our knowledge, GAAO offers a novel integration of event-sourced behaviour, hierarchical semantic structure, constraint governance, condition-space modelling, evidential reasoning, and multi-scale adaptation within a single agent ontology.
Existing frameworks such as BDI, ACT-R, RL, MAPE-K, and event-sourced system designs each address subsets of these concerns.
GAAO is intended as a unifying ontology that can host and interoperate with those approaches while providing an explicit temporal--semantic--constraint--evidential backbone.
\end{quote}

\section{Conclusion}

We have introduced GAAO and provided a formal specification of its core components.
By defining adaptive agents as tuples over event ledgers, semantic topologies, constraint fabrics, condition spaces, evidential graphs, transformation layers, adaptive reasoning engines, and recursive adaptation loops, GAAO offers a general ontology for modeling and steering adaptive behaviour over time.

Future work includes:
\begin{itemize}
    \item instantiating specific architectures derived from GAAO (e.g.\ cognitive agents, life-architecture systems, organisational agents),
    \item developing formal proofs about stability, convergence, or safety properties under specific instantiations of $\mathcal{L}$,
    \item empirical evaluation of GAAO-based architectures in real-world domains.
\end{itemize}

\end{document}
